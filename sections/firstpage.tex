\newpage
\begin{center} 
\huge Ограниченные метрики.\\ [10pt]
\vspace{\baselineskip}
\end{center} 
\textbf{Определение:} Если (X,d) - метрическое пространство, то можно задать \\новые метрики для пространства X: \\
\begin{center} 
$ \updelta = \frac{d}{(1+d)} $и $ \Updelta=min(d,1) $.
\end{center} 
\vspace{\baselineskip}
\par \textit{1)}  $\updelta$ будет метрикой, чтобы показать это нужно доказать неравенство треугольника:
\begin{center} 
$  \dfrac{d(x,y)}{1+d(x,y)}+\dfrac{d(y,z)}{1+d(y,z)} \geq \dfrac{d(x,y)+d(y,z)}{1+d(x,y)+d(y,z)} =$\\ 
\vspace{\baselineskip}
$=[[d(x,y)+d(y,z)]^{-1}+1]^{-1} \geq [[d(x,z)]^{-1}+1]^{-1}=\updelta(x,z).$
\end{center} 
\vspace{\baselineskip}

\par \textit{2)} Так как $d=\dfrac{\updelta}{1-\updelta}$, то метрика $\updelta$ эквивалентна метрике d, ведь каждый открытый шар одной содержит открытый шар другой.
\vspace{\baselineskip}

\par \textit{3)} Метрика $\updelta$  ограничена(1), но пространство $(X,\updelta)$ должно быть полностью ограниченным. Предположим, например, что  (X,d) - метрическое пространство  над $\mathds{R}$  с Евклидовой метрикой. Тогда $B_{\updelta}(x,\epsilon)=\{y| \updelta(x,y)<\epsilon \}=\{y|\,|x-y|<\frac{\epsilon}{1-\epsilon}\}=B_{d}(x,\frac{\epsilon}{1-\epsilon})$.Для достаточно маленького $\epsilon$ нет конечного набора шаров, которые могут покрыть $\mathds{R}$.
\vspace{\baselineskip}

\par \textit{4)} Путем повторения процесса, в котором мы выделяем метрику $\updelta$ из метрики d, можно получить последовательность эквивалентных ограниченных метрик $\{d_{n}\}$, заданные: $d_{n+1}=\frac{d_{n}}{1+d_{n}}$.
\vspace{\baselineskip}

\par \textit{5)} $\Updelta$ также ограничена единицей и является метрикой для X, ведь $ min(d(x,y),1) \leq$\\
 $\leq min(d(x,z)+d(y,z),1) \leq min (d(x,z),1)+min(d(z,y),1)$. Очевидно $\Updelta$ эквивалентна d.
 
 \par \textit{6)} Если (X,d) - метрическое над $\mathds{R}$ с Евклидовой метрикой, открытый шар: $B_{\Updelta}(0,1)$ является интервалом (-1,1). Его замыкание $\overline{B_{\Updelta}(0,1)}\, :$  [-1;1], а закрытый шар $\{ x|\, \Updelta(0,x) \leq1\}$ равен X.
 \vspace{\baselineskip}
 \newpage
  \par \textit{7)}  Каждая ограниченная метрика в топологическом пространстве X может быть использована, чтобы определить метрику Фреша на произведении топологических пространств $X^{z}=\prod\limits_{i = 1}^{\infty}X_{i}$(где $X_{i}=X$ для $\forall i=\overline{1,\infty} $), которая задаёт Тихоновское произведение топологических пространств.  Если $\updelta $ ограниченная метрика на X, мы определяем произведение топологических пространств как $d^{*}(x,y)=d^{*}(\left\langle x_{1},x_{2},... \right\rangle,\left\langle y_{1},y_{2},... \right\rangle)=\sum 2^{-i} \updelta(x_{i},y_{i})$. Доказывается, что топология $(X^{z},d^{*})$ является Тихоновским произведением топологических пространств через прямое сжатие базисных окрестностей.
  \vspace{\baselineskip}
  
    \par \textit{8)} Если (X,d) - метрическое пространство над $\mathds{R}$ с Евклидовой метрикой, то можно определить специальную метрику:
    \begin{center}
    $\upsigma(x,y)=|\frac{x}{1+|x|}-\frac{y}{1+|y|}|$.
    \end{center} 
   Героические, но напряженные вычисления позволяют убедиться, что $\upsigma$ является метрикой и что она определяет Евклидову топологию.  На самом деле  $\upsigma(x,y)<$\\ $<|x-y|$ для $\forall x,y \in X$. Но в $(X,\upsigma)$ положительные целые числа образуют последовательность Коши, так как $\upsigma(n,m)=\frac{|n-m|}{(1+|n|)(1+|m|)}$. Естественно эта последовательность не имеет предела в X, таким образом $(X,\upsigma)$не является полным метрическим пространством.\\
 \begin{center}
 \textbf{Дополнительные материалы:}\\
 \end{center} 
  \begin{center}
  \textbf{Определение:}\\ 
   \end{center} 
   \textit{Тихоновское произведение топологических пространств} - топологическое пространство, полученное, как множество, декартовым произведением исходных топологических пространств, топология которого задается с наложением ограничения, называемого тихоновской топологией произведения семейства топологических пространств.  Эта конструкция является произведением в категории всех топологических пространств, то есть для любой пары $(X, \{f_\alpha \})$ где $f_\alpha: X\to X_\alpha$ — отображение некоторого пространства X в пространства-сомножители, существует единственное отображение $f : X\to \prod_\alpha X_\alpha$
такое что для всех проекций $\pi_\alpha$ на пространства-сомножители верно $f_\alpha = \pi_\alpha \circ f$. В некотором смысле произведение пространств — это наиболее общее пространство, которое можно из них построить.
   \begin{center}
  \textbf{Определение:}\\ 
   \end{center} 
    Пусть $\{X_{\alpha}: \alpha\in A\}$ — семейство топологических пространств, A — индексное множество этого семейства, $X=\prod\limits_{\alpha\in A}X_{\alpha}$ — их декартово произведение, $\pi_{\alpha}: X\to X_{\alpha}$ — проекция произведения на соответствующий сомножитель, $\mathfrak{T}_{\alpha}$ —   \textit {топология} (множество всех открытых множеств)    \textit{пространства} $X_{\alpha}$.\\
 \textit{Тихоновская топология на произведении топологических пространств} — это минимальная топология, в которой все проекции на исходные пространства $\pi_{\alpha}$  непрерывны.\\
 Конструктивно её можно также описать следующим образом: в качестве предбазы топологии на X берётся семейство множеств $ \mathfrak{P}=\{\pi_{\alpha}^{-1}(U): \alpha\in A,\, U\in \mathfrak{T}_{\alpha}\}.$  \textit{База топологии} — всевозможные конечные пересечения множеств из $\mathfrak{P}$, а \textit{топология} — всевозможные объединения множеств из базы.\\
 Отметим, что тихоновская топология является гораздо более слабой, чем несколько более естественная «коробочная» топология, для которой базу топологии образуют всевозможные произведения открытых подмножеств перемножаемых пространств. Такая топология не обладает указанным выше свойством универсальности и для неё не верна теорема Тихонова.
 \begin{center}
 \textbf{Теорема Тихонова:}
  \end{center} 
     \textit{Если все множества} $X_{\alpha}$    \textit{компактны, тогда компактно и их тихоновское произведение}.\\
   \textbf{Доказательство:} Согласно теореме Александера о предбазе, достаточно доказать, что всякое покрытие элементами предбазы $\mathfrak{P}$ допускает конечное подпокрытие. Для всякого $\alpha$ пусть $V_{\alpha}$ — объединение всех множеств $U\in X_{\alpha}$, для которых множество $\pi_{\alpha}^{-1}(U) $содержится в покрытии. Тогда непокрытая часть пространства X, выражается формулой
$\prod\limits_{\alpha\in A}X_{\alpha}\setminus V_{\alpha}$.
Поскольку это множество пусто, пустым должен быть хотя бы один сомножитель. Это означает, что рассматриваемое покрытие при некотором $\alpha$ содержит $\pi_{\alpha}$ - прообраз покрытия пространства $X_{\alpha}$. В силу компактности пространства $X_{\alpha}$, из его покрытия можно выделить конечное подпокрытие, и тогда его прообраз относительно отображения $\pi_{\alpha}$ будет конечным подпокрытием пространства X.
  